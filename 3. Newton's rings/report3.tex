% chktex-file 44

\documentclass[a4paper,12pt]{article}
\usepackage[utf8]{inputenc}
\usepackage[russian]{babel}
\usepackage{geometry}
\usepackage{amsthm}
\usepackage[dvipsnames]{xcolor}
\usepackage{framed}
\usepackage{booktabs}
\usepackage{array}
\usepackage{amssymb}
\usepackage{adjustbox}
\usepackage{makecell}
\usepackage{float}
\usepackage{graphicx}

\definecolor{shadecolor}{RGB}{245,245,247} 
\geometry{left=2cm, right=2cm, top=2cm, bottom=2cm}

\title{Модель №.3. \\ Оптика. Кольца Ньютона }
\author{Ким В.Р., Вишневский С.А \\ Группа M3207 }
\date{}

\theoremstyle{definition}
\newtheorem*{task}{Задание}\setlength{\parindent}{0pt}

\newenvironment{solution}
{\begin{shaded}\textbf{Решение:}\par\setlength{\parindent}{0pt}}
{\end{shaded}}

\newenvironment{answer}
{\par\noindent\textbf{Ответ:} }
{\par}

\begin{document}
\maketitle

\begin{task}
    Моделирование колец Ньютона для линзы заданного радиуса. 
    Рассмотреть монохроматический и квазимонохроматический свет 
    (задается середина и ширина спектра в нанометрах). Вывод цветного 
    распределения интенсивности интерференционной картины и графика
    зависимости интенсивности от радиальной координаты.

\end{task}


\end{document}