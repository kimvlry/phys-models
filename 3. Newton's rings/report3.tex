% chktex-file 44

\documentclass[a4paper,11pt]{article}
\usepackage[utf8]{inputenc}
\usepackage[russian]{babel}
\usepackage{geometry}
\usepackage{amsthm}
\usepackage[dvipsnames]{xcolor}
\usepackage{framed}
\usepackage{booktabs}
\usepackage{array}
\usepackage{amssymb}
\usepackage{adjustbox}
\usepackage{makecell}
\usepackage{float}
\usepackage{graphicx}
\usepackage{amsmath}
\usepackage{physics}
\usepackage{hyperref}

\definecolor{shadecolor}{RGB}{245,245,247} 
\geometry{left=2cm, right=2cm, top=2cm, bottom=2cm}

\title{Модель №.3. \\ Оптика. Кольца Ньютона }
\author{Ким В.Р., Вишневский С.А \\ Группа M3207 }
\date{}

\theoremstyle{definition}
\newtheorem*{task}{Задание}\setlength{\parindent}{0pt}

\newenvironment{solution}
{\begin{shaded}\textbf{Решение:}\par\setlength{\parindent}{0pt}}
{\end{shaded}}

\newenvironment{answer}
{\par\noindent\textbf{Ответ:} }
{\par}

\begin{document}
\maketitle

\begin{task}
    Моделирование колец Ньютона для линзы заданного радиуса. 
    Рассмотреть монохроматический и квазимонохроматический свет 
    (задается середина и ширина спектра в нанометрах). Вывод цветного 
    распределения интенсивности интерференционной картины и графика
    зависимости интенсивности от радиальной координаты.
\end{task}





\subsection*{1. Введение}
Кольца Ньютона — это интерференционная картина в виде концентрических светлых 
и тёмных колец, наблюдаемая при освещении системы <<выпуклая линза на плоской пластине>> светом. 
Возникает она в результате интерференции отражённого света в тонком воздушном зазоре между 
поверхностями линзы и пластины. Толщина зазора зависит от расстояния от точки касания, 
поэтому фаза отражённого света изменяется по радиусу, формируя характерные кольца.

Целью моделирования является вычисление и визуализация распределения интенсивности отражённого света 
в зависимости от радиальной координаты, как для монохроматического, так и для квазимонохроматического 
освещения.



\subsection*{2. Геометрия системы}
Рассматривается выпуклая линза с радиусом кривизны \( R \), которая лежит 
на плоской стеклянной пластине. Между ними образуется тонкий воздушный зазор переменной 
толщины. Вблизи центра (при \( h \ll R \)) толщина зазора \( h(r) \) описывается:

\[
h(r) = \frac{r^2}{2R}
\]

где:
\begin{itemize}
  \item \( r \) — расстояние от центра контактной точки (в метрах),
  \item \( R \) — радиус кривизны линзы (в метрах),
  \item \( h(r) \) — толщина воздушного слоя.
\end{itemize}



\subsection*{3. Интерференция света}
Свет отражается от двух границ: верхней (линза–воздух) и нижней (воздух–пластина). 
В результате возникают две когерентные волны, интерферирующие между собой.



\subsubsection*{3.1 Разность хода}
Разность оптических путей между двумя отражёнными волнами:

\[
\Delta = 2h(r) + \frac{\lambda}{2}
\]

где \( \lambda \) — длина волны света, а \( \lambda/2 \) — поправка на сдвиг фазы при отражении от более плотной среды.

\subsubsection*{3.2 Условия интерференции}

Для тёмных колец (минимум интенсивности):

\[
2h(r) = (2m + 1)\frac{\lambda}{2}, \quad m = 0, 1, 2, \dots
\]
\[
r_m = \sqrt{(2m + 1)\frac{\lambda R}{2}}
\]

Для светлых колец (максимум интенсивности):

\[
2h(r) = m \lambda
\]
\[
r_m = \sqrt{m \lambda R}
\]

где \( m \) — порядок интерференции.



\subsection*{4. Интенсивность отражённого света}
Интенсивность отражённого света рассчитывается как:

\[
I(r) = I_0 \cdot \left[1 + \cos\left(\frac{2\pi \Delta}{\lambda} \right)\right]
\]

Подставляя разность хода:

\[
I(r) = I_0 \cdot \left[1 + \cos\left(\frac{4\pi h(r)}{\lambda} + \pi \right)\right]
\]

С учётом \( \cos(x + \pi) = -\cos(x) \):

\[
I(r) = I_0 \cdot \left[1 - \cos\left(\frac{4\pi h(r)}{\lambda} \right)\right]
\]

И, подставляя \( h(r) \):

\[
I(r) = I_0 \cdot \left[1 - \cos\left(\frac{2\pi r^2}{\lambda R} \right)\right]
\]



\subsection*{5. Квазимонохроматический свет}
При освещении светом с конечной спектральной шириной \( \Delta\lambda \), свет состоит из диапазона 
длин волн \( \lambda_i \in [\lambda_0 - \Delta\lambda/2, \lambda_0 + \Delta\lambda/2] \). 
Интенсивность отражения тогда определяется усреднением по спектру:

\[
I(r) = \frac{1}{N} \sum_{i=1}^{N} \left[1 - \cos\left( \frac{2\pi r^2}{\lambda_i R} \right) \right] \cdot S(\lambda_i)
\]

где \( S(\lambda) \) — спектральная плотность:

\[
S(\lambda) = \exp\left( -\frac{(\lambda - \lambda_0)^2}{2\sigma^2} \right), \quad \sigma = \frac{\Delta\lambda}{2\sqrt{2 \ln 2}}
\]



\subsection*{6. Цветовое изображение интерференционной картины}
Для двумерной визуализации кольцевой структуры строится двумерное поле интенсивности:

\[
I(x, y) = I\left( \sqrt{x^2 + y^2} \right)
\]

В случае квазимонохроматического света для каждой длины волны \( \lambda_i \) создаётся изображение 
и суммируется с соответствующим весом. Полученный спектр в каждой точке переводится в RGB с помощью 
соответствующей модели цветового зрения.



\subsection*{7. Итоги}
На основе этой теории можно:
\begin{itemize}
    \item Рассчитать радиусы тёмных и светлых колец;
    \item Построить зависимость интенсивности от расстояния \( r \);
    \item Получить цветное изображение интерференционной картины;
    \item Исследовать влияние спектральной ширины на чёткость колец.
\end{itemize}

\end{document}
