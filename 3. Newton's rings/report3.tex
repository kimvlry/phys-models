% chktex-file 44

\documentclass[a4paper,11pt]{article}
\usepackage[utf8]{inputenc}
\usepackage[russian]{babel}
\usepackage{geometry}
\usepackage{amsthm}
\usepackage[dvipsnames]{xcolor}
\usepackage{framed}
\usepackage{booktabs}
\usepackage{array}
\usepackage{amssymb}
\usepackage{adjustbox}
\usepackage{makecell}
\usepackage{float}
\usepackage{graphicx}
\graphicspath{{../}}
\usepackage{amsmath}
\usepackage{physics}
\usepackage{hyperref}
\usepackage{color}
\usepackage{listings}

\definecolor{shadecolor}{RGB}{245,245,247}
\geometry{left=2cm, right=2cm, top=2cm, bottom=2cm}

\title{Модель №.3. \\ Оптика. Кольца Ньютона }
\author{Ким В.Р., Вишневский С.А \\ Группа M3207 }
\date{}

\theoremstyle{definition}
\newtheorem*{task}{Задание}\setlength{\parindent}{0pt}

\newenvironment{solution}
{\begin{shaded}
     \textbf{Решение:}\par\setlength{\parindent}{0pt}}
     {
\end{shaded}}

\newenvironment{answer}
{\par\noindent\textbf{Ответ:} }
{\par}

\definecolor{codegreen}{rgb}{0,0.6,0}
\definecolor{codegray}{rgb}{0.5,0.5,0.5}
\definecolor{codepurple}{rgb}{0.58,0,0.82}
\definecolor{backcolour}{rgb}{0.95,0.95,0.92}

\lstdefinelanguage{MyPython}{
    language=Python,
    morekeywords={np, plt, self, cos, sin},
    sensitive=true,
}
\lstdefinestyle{mystyle}{
    backgroundcolor=\color{backcolour},
    commentstyle=\color{codegreen},
    keywordstyle=\color{magenta},
    numberstyle=\tiny\color{codegray},
    stringstyle=\color{codepurple},
    basicstyle=\ttfamily\footnotesize,
    breakatwhitespace=false,
    breaklines=true,
    captionpos=b,
    keepspaces=true,
    numbers=left,
    numbersep=5pt,
    showspaces=false,
    showstringspaces=false,
    showtabs=false,
    tabsize=2,
    columns=fullflexible,
}

\lstset{style=mystyle}

\begin{document}
    \maketitle

    \begin{task}
        Моделирование колец Ньютона для линзы заданного радиуса.
        Рассмотреть \textit{монохроматический} и \textit{квазимонохроматический} свет
        (задается середина и ширина спектра в нанометрах). Вывод цветного
        распределения интенсивности интерференционной картины и графика
        зависимости интенсивности от радиальной координаты.
    \end{task}


    \section{Теория}

    \subsection{Введение}
    Кольца Ньютона — это интерференционная картина в виде концентрических светлых
    и тёмных колец, наблюдаемая при освещении системы <<выпуклая линза на плоской пластине>> светом.
    Возникает она в результате \textit{интерференции}*** отражённого света в тонком воздушном зазоре между
    поверхностями линзы и пластины. Толщина зазора зависит от расстояния от точки касания,
    поэтому фаза отражённого света изменяется по радиусу, формируя характерные кольца.

    Целью моделирования является вычисление и визуализация распределения интенсивности отражённого света
    в зависимости от радиальной координаты, как для \textit{монохроматического}**, так и для
    \textit{квазимонохроматического}** освещения.
    \\
    \\
    * \textbf{интерференция} — взаимное увеличение или уменьшение результирующей амлитуды двух или нескольких
    когерентных**** волн при их наложении друг на друга
    \\
    \\
    ** \textbf{монохроматический свет} — это световые колебания одной частоты.
    \\
    \\
    *** \textbf{квазихроматический свет} можно представить как суперпозицию монохроматических волн, частоты которых
    расположены в узком спектральном диапазоне.
    \\
    \\
    **** \textbf{когерентные волны:} если частоты колебаний в обеих волнах одинаковы, а разность фаз возбуждаемых колебаний
    остается постоянной во времени, то такие волны называются когерентными.

    \subsection{Геометрия системы}
    Рассматривается выпуклая линза с радиусом кривизны \( R \), которая лежит
    на плоской стеклянной пластине. Между ними образуется тонкий воздушный зазор переменной
    толщины. Вблизи центра (при \( h \ll R \)) толщина зазора \( h(r) \) описывается:
    \begin{equation}
        h(r) = \frac{r^2}{2R}\label{eq:equation}
    \end{equation}

    где:
    \begin{itemize}
        \item \( r \) — расстояние от центра контактной точки (в метрах),
        \item \( R \) — радиус кривизны линзы (в метрах),
        \item \( h(r) \) — толщина воздушного слоя.
    \end{itemize}

    \subsection{Интерференция света}
    Свет отражается от двух границ: верхней (линза–воздух) и нижней (воздух–пластина).
    В результате возникают две когерентные волны, интерферирующие между собой.

    \subsubsection{Разность хода}
    Разность оптических путей между двумя отражёнными волнами:
    \begin{equation}
        \Delta = 2h(r) + \frac{\lambda}{2}\label{eq:equation2}
    \end{equation}

    где \( \lambda \) — длина волны света, а \( \lambda/2 \) — поправка на сдвиг фазы при отражении от более плотной среды.

    \subsubsection{Условия интерференции}

    Для тёмных колец (минимум интенсивности):

    \begin{gather}
        2h(r) = (2m + 1)\frac{\lambda}{2}, \quad m = 0, 1, 2, \dots\\
        r_m = \sqrt{(2m + 1)\frac{\lambda R}{2}}\\
    \end{gather}

    Для светлых колец (максимум интенсивности):

    \begin{gather}
        2h(r) = m \lambda\\
        r_m = \sqrt{m \lambda R}\\
    \end{gather}

    где \( m \) — порядок интерференции.

    \subsection{Интенсивность отражённого света}
    Интенсивность отражённого света рассчитывается как:

    \begin{equation}
        I(r) = I_0 \cdot \left[1 + \cos\left(\frac{2\pi \Delta}{\lambda} \right)\right]\label{eq:equation3}
    \end{equation}

    Подставляя разность хода:

    \begin{equation}
        I(r) = I_0 \cdot \left[1 + \cos\left(\frac{4\pi h(r)}{\lambda} + \pi \right)\right]\label{eq:equation4}
    \end{equation}

    С учётом \( \cos(x + \pi) = -\cos(x) \):

    \begin{equation}
        I(r) = I_0 \cdot \left[1 - \cos\left(\frac{4\pi h(r)}{\lambda} \right)\right]\label{eq:equation5}
    \end{equation}

    И, подставляя \( h(r) \):

    \begin{equation}
        I(r) = I_0 \cdot \left[1 - \cos\left(\frac{2\pi r^2}{\lambda R} \right)\right]\label{eq:equation6}
    \end{equation}

    \subsection{Квазимонохроматический свет}
    При освещении светом с конечной спектральной шириной \( \Delta\lambda \), свет состоит из диапазона
    длин волн, распределённых вокруг центрального значения \(\lambda_0\) : \( \lambda_i \in [\lambda_0 - \Delta\lambda/2, \lambda_0 + \Delta\lambda/2] \).
    Интенсивность отражения тогда определяется усреднением по спектру:

    \begin{equation}
        I(r) = \frac{1}{N} \sum_{i=1}^{N} \left[1 - \cos\left( \frac{2\pi r^2}{\lambda_i R} \right) \right] \cdot S(\lambda_i)\label{eq:equation7}
    \end{equation}

    где \( S(\lambda) \) - вклад каждой длины волны \(\lambda\), он определяется спектральной плотностью,
    распределенной по Гауссу:

    \begin{equation}
        S(\lambda) = \exp\left( -\frac{(\lambda - \lambda_0)^2}{2\sigma^2} \right), \quad \sigma = \frac{\Delta\lambda}{2\sqrt{2 \ln 2}}\label{eq:equation8}
    \end{equation}

    \subsection{Цветовое изображение интерференционной картины}
    Для двумерной визуализации кольцевой структуры строится двумерное поле интенсивности:

    \begin{equation}
        I(x, y) = I\left( \sqrt{x^2 + y^2} \right)\label{eq:equation9}
    \end{equation}

    \subsection{Итоги}
    На основе этой теории можно:
    \begin{itemize}
        \item Рассчитать радиусы тёмных и светлых колец;
        \item Построить зависимость интенсивности от расстояния \( r \);
        \item Получить цветное изображение интерференционной картины;
        \item Исследовать влияние спектральной ширины на чёткость колец.
    \end{itemize}

%   ----------------------------------

    \newpage


    \section{Моделирование}

    В данном разделе описывается структура программы, реализующей моделирование колец Ньютона

    \subsection{Пользовательский интерфейс и настройка параметров}
    В этой части кода реализован графический интерфейс пользователя на базе \texttt{Tkinter}.
    Интерфейс позволяет задать следующие параметры:
    \begin{itemize}
        \item Радиус кривизны \( R \) (м).
        \item Центральная длина волны \( \lambda_0 \) (нм), которая затем переводится в метры.
        \item Ширина спектра \(\Delta \lambda\) (нм), также переводимая в метры.
        \item Режим освещения - моно-, квази-хроматический, \textit{белый свет}
    \end{itemize}
    Под \emph{белым светом} в данной работе понимается непрерывный спектр видимого диапазона,
    состоящий из волн разных длин, объединённых без выраженного пикового распределения.
    В упрощённой модели этот спектр аппроксимируется равномерным распределением длин волн
    от 400\,нм до 700\,нм.

    \subsection{Расчёт интенсивности}
    Программа рассчитывает интенсивность интерференционной картины в зависимости от радиальной координаты \( r \).
    Для этого используются две функции:
    \begin{itemize}
        \item \textbf{Монохроматический свет:} Функция \texttt{intensity\_mono} рассчитывает интенсивность
        по формуле (9), где \(I_0\) представлена коэффициентом нормировки (в коде установлено значение 0.5).
        Значение выражения \(1 - \cos(\cdot)\) изменяется от 0 до 2, поскольку
        \(\
        \cos\left(\frac{2\pi r^2}{\lambda R}\right) \in [-1, 1].
        \)
        Если принять \(I_0 = 0.5\), то получим
        \[
            I(r) = 0.5\left[1 - \cos\left(\frac{2\pi r^2}{\lambda R}\right)\right],
        \]
        что нормирует интенсивность в диапазон [0, 1]. Такой выбор удобен для визуализации, так как после масштабирования
        (например, перемножением на 255) можно получить корректное 8-битное изображение, где 0 соответствует полной
        темноте, а 255 --- максимальной яркости.

        \begin{lstlisting}[language=MyPython, label={lst:lstlisting}]
        def intensity_mono(self, r, wavelength):
            return 0.5 * (1 - np.cos(2 * np.pi * r ** 2 / (wavelength * self.R)))
        \end{lstlisting}

        \item{\textbf{Квазимонохроматический случай. Усреднение с использованием спектральной плотности}}
        При квазимонохроматическом освещении свет рассматривается как диапазон длин волн
        \[
            \lambda_i \in \left[\lambda_0 - \frac{\Delta\lambda}{2}, \, \lambda_0 + \frac{\Delta\lambda}{2}\right],
        \]
        распределённых вокруг центрального значения \(\lambda_0\). Теоретически интенсивность
        определяется по формуле (13), где \(S(\lambda)\) задаётся Гауссовой функцией (14)

        В реализации используется следующий подход:
        \begin{itemize}
            \item Генерируется массив из 20 равномерных длин волн \(\{\lambda_i\}\) в указанном диапазоне.
            \item Вычисляются веса
            \[
                w_i = \frac{S(\lambda_i)}{\sum_{j=1}^{20} S(\lambda_j)},
            \]
            с помощью одной векторной операции:
            \texttt{weights = np.array([self.spectral\_density(wl) for wl in wavelengths]); weights /= weights.sum()}.
            \item Создаётся трёхканальный массив \(\mathrm{image}[x,y,k]\), \(k\in\{0,1,2\}\) для RGB:
            \begin{itemize}
                \item для каждой \(\lambda_i\) рассчитывается скалярное поле интенсивности
                \(I_{\rm mono}(r,\lambda_i)\) функцией \texttt{intensity\_mono},
                \item находят цветовые коэффициенты \(\mathrm{rgb} = \texttt{wavelength\_to\_rgb}(\lambda_i)\),
                \item и суммируют вклад во все три канала:
                \[
                    \mathrm{image}[:,:,k] \;\mathrel{+}=\, w_i\,I_{\rm mono}(r,\lambda_i)\,\mathrm{rgb}[k].
                \]
            \end{itemize}
            \item После цикла массив нормализуется по максимуму,
            умножается на 255 и приводится к типу \texttt{uint8}:
            \[
                \mathrm{image} \;\to\; \frac{\mathrm{image}}{\max(\mathrm{image})}\times255.
            \]
        \end{itemize}

        \begin{lstlisting}[language=MyPython,label={lst:lstlisting2}]
            def intensity_quasi(self, r):
                wavelengths = np.linspace(self.lambda0 - self.delta_lambda / 2,
                                          self.lambda0 + self.delta_lambda / 2, 20)
                weights = np.array([self.spectral_density(wl) for wl in wavelengths])
                weights /= np.sum(weights)

                image = np.zeros((r.shape[0], r.shape[1], 3))
                for wl, weight in zip(wavelengths, weights):
                    I = self.intensity_mono(r, wl)
                    rgb = wavelength_to_rgb(wl)
                    for i in range(3):
                        image[:, :, i] += weight * I * rgb[i]
                image /= np.max(image)
                return (image * 255).astype(np.uint8)
        \end{lstlisting}

        Итого, \texttt{intensity\_quasi} возвращает готовое RGB-изображение. Пусть через
        \(\mathrm{rgb}_k(\lambda_i)\), \(k\in\{R,G,B\}\), обозначается \(k\)-я компонентa,
        возвращаемая функцией \texttt{wavelength\_to\_rgb}(\(\lambda_i\)).
        Тогда для каждого канала выполняется взвешенная сумма:
        \[
            \begin{aligned}
                R(r) &= \sum_{i=1}^{20} w_i \; I_{\rm mono}(r,\lambda_i)\;\mathrm{rgb}_R(\lambda_i),\\
                G(r) &= \sum_{i=1}^{20} w_i \; I_{\rm mono}(r,\lambda_i)\;\mathrm{rgb}_G(\lambda_i),\\
                B(r) &= \sum_{i=1}^{20} w_i \; I_{\rm mono}(r,\lambda_i)\;\mathrm{rgb}_B(\lambda_i),
            \end{aligned}
        \]
        где
        \[
            w_i = \frac{S(\lambda_i)}{\sum_{j=1}^{20}S(\lambda_j)},
            \qquad
            I_{\rm mono}(r,\lambda_i) = 0.5\bigl[1 - \cos\!(2\pi r^2/(\lambda_i R))\bigr].
        \]
        После вычисления всех трёх каналов массив \(\bigl[R(r),G(r),B(r)\bigr]\) нормализуется
        по своему максимуму и масштабируется в диапазон \([0,255]\), получая итоговое 8-битное RGB-изображение.

    \end{itemize}

    \subsection{Построение двумерного поля интенсивности}
    Для визуализации интерференционной картины создаётся двумерная координатная сетка, где интенсивность \( I(x,y) \)
    определяется как функция \( I\left(\sqrt{x^2+y^2}\right) \). Это соответствует преобразованию теоретической
    зависимости \( I(r) \) в изображение, что позволяет получить кольцевую структуру модели.

    \begin{lstlisting}[language=MyPython,label={lst:lstlisting3}]
        x = np.linspace(-0.01, 0.01, self.size)
        y = np.linspace(-0.01, 0.01, self.size)
        xx, yy = np.meshgrid(x, y)
        r = np.sqrt(xx ** 2 + yy ** 2)
    \end{lstlisting}


    \section{Демо-примеры запуска программы}

    \subsection{Монохром, близкий к ИК}
    \includegraphics[scale=0.7]{3. Newton's rings/demo results/infraRed_mono}

    \subsection{Монохром, близкий к УФ}
    \includegraphics[scale=0.7]{3. Newton's rings/demo results/ultraViolet_mono}

    \subsection{Зеленый монохром}
    \includegraphics[scale=0.7]{3. Newton's rings/demo results/green_mono}

    \subsection{Большой квазихром}
    \includegraphics[scale=0.7]{3. Newton's rings/demo results/big_quasi}

    \subsection{Маленький квазихром}
    \includegraphics[scale=0.7]{3. Newton's rings/demo results/small_quasi}

    \subsection{Квазихром с широким спектром волн}
    \includegraphics[scale=0.7]{3. Newton's rings/demo results/wideSpectre_quasi}

    \subsection{Белый свет}
    \includegraphics[scale=0.7]{3. Newton's rings/demo results/whiteLight}


    \section{Вывооооод}
    Можно заметить, что интерференция на определенных промежутках становится невидимой, а потом (с большим радиусом) возвращается.
    Почему?(?)
    Для тёмных колец (минимумов интерференции) радиус $r_m$ определяется выражением:
    \[
        r_m = \sqrt{(2m + 1) \frac{\lambda R}{2}}
    \]
    где:
    \begin{itemize}
        \item $m$ — порядок минимума ($m = 0, 1, 2, \ldots$),
        \item $\lambda$ — длина волны света,
        \item $R$ — радиус кривизны линзы.
    \end{itemize}

    Для светлых колец (максимумов интерференции):
    \[
        r_m = \sqrt{m \lambda R}
    \]

    \textbf{Причины исчезновения интерференции}
    Свет содержит множество длин волн $\lambda$ (от 400 до 700~нм).
    Каждая длина волны создаёт собственную систему колец с различными радиусами.
    В результате:

    Кольца от разных длин волн \textbf{не совпадают} пространственно:
    \[
        r_m^{(\lambda_1)} = \sqrt{m \lambda_1 R}, \quad
        r_m^{(\lambda_2)} = \sqrt{m \lambda_2 R}
    \]

    При наложении разных систем колец происходит \textbf{взаимное гашение}: свет одного цвета может находиться в максимуме,
    другой — в минимуме.

    \textbf{Причины восстановления интерференции}
    При больших номерах $m$ может наблюдаться частичное совпадение колец разных длин волн.
    Это возможно, если для двух длин волн $\lambda_1$ и $\lambda_2$ выполняется:

    \[
        \sqrt{m_1 \lambda_1 R} \approx \sqrt{m_2 \lambda_2 R} \quad \Rightarrow \quad m_1 \lambda_1 \approx m_2 \lambda_2
    \]

    Таким образом, кольца от разных длин волн снова накладываются конструктивно (потому что волны усиливают друг друга),
    и картина интерференции \textbf{восстанавливается}.

\end{document}
